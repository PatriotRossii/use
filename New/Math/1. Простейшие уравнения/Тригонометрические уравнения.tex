% !TEX TS-program = pdflatex
% !TEX encoding = UTF-8 Unicode

% This is a simple template for a LaTeX document using the "article" class.
% See "book", "report", "letter" for other types of document.

\documentclass[11pt]{article} % use larger type; default would be 10pt

\usepackage[utf8]{inputenc} % set input encoding (not needed with XeLaTeX)
\usepackage[english,russian]{babel}

\usepackage{amsmath}

%%% Examples of Article customizations
% These packages are optional, depending whether you want the features they provide.
% See the LaTeX Companion or other references for full information.

%%% PAGE DIMENSIONS
\usepackage{geometry} % to change the page dimensions
\geometry{a4paper} % or letterpaper (US) or a5paper or....
% \geometry{margin=2in} % for example, change the margins to 2 inches all round
% \geometry{landscape} % set up the page for landscape
%   read geometry.pdf for detailed page layout information

\usepackage{graphicx} % support the \includegraphics command and options

% \usepackage[parfill]{parskip} % Activate to begin paragraphs with an empty line rather than an indent

%%% PACKAGES
\usepackage{booktabs} % for much better looking tables
\usepackage{array} % for better arrays (eg matrices) in maths
\usepackage{paralist} % very flexible & customisable lists (eg. enumerate/itemize, etc.)
\usepackage{verbatim} % adds environment for commenting out blocks of text & for better verbatim
\usepackage{subfig} % make it possible to include more than one captioned figure/table in a single float
% These packages are all incorporated in the memoir class to one degree or another...

%%% HEADERS & FOOTERS
\usepackage{fancyhdr} % This should be set AFTER setting up the page geometry
\pagestyle{fancy} % options: empty , plain , fancy
\renewcommand{\headrulewidth}{0pt} % customise the layout...
\lhead{}\chead{}\rhead{}
\lfoot{}\cfoot{\thepage}\rfoot{}

%%% SECTION TITLE APPEARANCE
\usepackage{sectsty}
\allsectionsfont{\sffamily\mdseries\upshape} % (See the fntguide.pdf for font help)
% (This matches ConTeXt defaults)

%%% ToC (table of contents) APPEARANCE
\usepackage[nottoc,notlof,notlot]{tocbibind} % Put the bibliography in the ToC
\usepackage[titles,subfigure]{tocloft} % Alter the style of the Table of Contents
\renewcommand{\cftsecfont}{\rmfamily\mdseries\upshape}
\renewcommand{\cftsecpagefont}{\rmfamily\mdseries\upshape} % No bold!

%%% END Article customizations

%%% The "real" document content comes below...

\title{Тригонометрические уравнения}
\author{Lisid Lakonsky}
%\date{} % Activate to display a given date or no date (if empty),
         % otherwise the current date is printed 

\begin{document}
\maketitle

\section{№ \textbf{77376}}

$$\tg \frac{\pi x}{4} = -1$$
$$\frac{\pi x}{4} = \arctg (-1) + \pi k, k \in Z$$
$$\frac{\pi x}{4} = -\frac{\pi}{4} + \pi k, k \in Z$$
$$\pi x = -\pi + 4\pi k, k \in Z$$
$$x = -1 + 4k, k \in Z$$

Подберем наибольший отрицательный корень:

Если $k = 0$, то $x = -1 + 4 * 0 = -1$

Если $k = 1$, то $x = -1 + 4 * 1 = 3$

Ответ: \textbf{-1}

\section{№ \textbf{26669}}

$$\cos \frac{\pi(x - 7)}{3} = \frac{1}{2}$$
$$\frac{\pi(x - 7)}{3} = \pm \arccos \frac{1}{2} + 2 \pi k, k \in Z$$
$$\frac{\pi(x - 7)}{3} = \pm \frac{\pi}{3} + 2 \pi k, k \in Z$$
$$\pi(x - 7) = \pm \pi + 6 \pi k, k \in Z$$
$$x - 7 = \pm 1 + 6k, k \in Z$$
$$x = \pm 1 + 6k + 7, k \in Z$$

Подберем наибольший отрицательный корень:

Если $k = 0$, то $x_1 = 1 + 0 + 7 = 8$, $x_2 = -1 + 6 + 7 = 12$

Если $k = -1$, то $x_1 = 1 - 6 + 7 = 2$, $x_2 = -1 - 6 + 7 = 0$

Если $k = -2$, то $x_1 = 1 - 12 + 7 = -4$, $x_2 = -1 - 12 + 7 = -6$

Ответ: \textbf{-4}

\section{№ \textbf{77377}}

$$\sin \frac{\pi x}{3} = 0.5$$
$$\frac{\pi x}{3} = (-1)^k \frac{\pi}{6} + \pi k, k \in Z$$
$$\pi x = (-1)^k \frac{\pi}{2} + 3 \pi k, k \in Z$$
$$x = (-1)^k \frac{1}{2} + 3k, k \in Z$$

Подберем наименьший положительный корень

Если $k = 0$, то $x = \frac{1}{2}$

Если $k = 1$, то $x = -\frac{1}{2} + 3 = 2.5$

Ответ: \textbf{0.5}

\end{document}
