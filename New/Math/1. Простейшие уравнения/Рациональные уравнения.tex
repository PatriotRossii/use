% !TEX TS-program = pdflatex
% !TEX encoding = UTF-8 Unicode

% This is a simple template for a LaTeX document using the "article" class.
% See "book", "report", "letter" for other types of document.

\documentclass[11pt]{article} % use larger type; default would be 10pt

\usepackage[utf8]{inputenc} % set input encoding (not needed with XeLaTeX)
\usepackage[english,russian]{babel}

\usepackage{amsmath}

%%% Examples of Article customizations
% These packages are optional, depending whether you want the features they provide.
% See the LaTeX Companion or other references for full information.

%%% PAGE DIMENSIONS
\usepackage{geometry} % to change the page dimensions
\geometry{a4paper} % or letterpaper (US) or a5paper or....
% \geometry{margin=2in} % for example, change the margins to 2 inches all round
% \geometry{landscape} % set up the page for landscape
%   read geometry.pdf for detailed page layout information

\usepackage{graphicx} % support the \includegraphics command and options

% \usepackage[parfill]{parskip} % Activate to begin paragraphs with an empty line rather than an indent

%%% PACKAGES
\usepackage{booktabs} % for much better looking tables
\usepackage{array} % for better arrays (eg matrices) in maths
\usepackage{paralist} % very flexible & customisable lists (eg. enumerate/itemize, etc.)
\usepackage{verbatim} % adds environment for commenting out blocks of text & for better verbatim
\usepackage{subfig} % make it possible to include more than one captioned figure/table in a single float
% These packages are all incorporated in the memoir class to one degree or another...

%%% HEADERS & FOOTERS
\usepackage{fancyhdr} % This should be set AFTER setting up the page geometry
\pagestyle{fancy} % options: empty , plain , fancy
\renewcommand{\headrulewidth}{0pt} % customise the layout...
\lhead{}\chead{}\rhead{}
\lfoot{}\cfoot{\thepage}\rfoot{}

%%% SECTION TITLE APPEARANCE
\usepackage{sectsty}
\allsectionsfont{\sffamily\mdseries\upshape} % (See the fntguide.pdf for font help)
% (This matches ConTeXt defaults)

%%% ToC (table of contents) APPEARANCE
\usepackage[nottoc,notlof,notlot]{tocbibind} % Put the bibliography in the ToC
\usepackage[titles,subfigure]{tocloft} % Alter the style of the Table of Contents
\renewcommand{\cftsecfont}{\rmfamily\mdseries\upshape}
\renewcommand{\cftsecpagefont}{\rmfamily\mdseries\upshape} % No bold!

%%% END Article customizations

%%% The "real" document content comes below...

\title{Рациональные уравнения}
\author{Lisid Lakonsky}
%\date{} % Activate to display a given date or no date (if empty),
         % otherwise the current date is printed 

\begin{document}
\maketitle

\section{№ \textbf{26664}}

$$ \frac{x - 119}{x + 7} = -5 $$
$$ x - 119 = -5x - 35 $$
$$ 6x = -35 + 119 $$
$$ 6x = 84 $$
$$ x = 14 $$

ОДЗ:

$$ x + 7 != 0 $$
$$ x != -7 $$

Ответ: \textbf{14}

\section{№ \textbf{26665}}

$$ x = \frac{6x - 15}{x - 2} $$
$$ x^2 - 2x = 6x - 15 $$
$$ x^2 = 8x - 15 $$
$$ x^2 - 8x + 15 = 0 $$
$$ D = 64 - (4 * 1 * 15) = 64 - 60 = 4 $$
$$ x_{12} = \frac{8 \pm 2}{2} = [5, 3 $$

ОДЗ:

$$ x - 2 \neq 0 $$
$$ x \neq 2 $$

Ответ: \textbf{5}

\section{№ \textbf{77366}}

$$\frac{9}{x^2 - 16} = 1$$
$$9 = x^2 - 16$$
$$x^2 - 25 = 0$$
$$(x - 5)(x + 5) = 0$$

Произведение равно нулю когда один из множителей равен нулю. Следовательно,

$$x - 5 = 0$$
$$x_1 = 5$$


$$x + 5 = 0$$
$$x_2 = -5$$

ОДЗ:

$$x_2 - 16 \neq 0$$
$$x_2 \neq 16$$
$$x \neq \pm 4$$

Ответ: \textbf{5}

\section{№ \textbf{77367}}

$$\frac{13x}{2x^2 - 7} = 1$$
$$13x = 2x^2 - 7$$
$$-2x^2 + 13x + 7 = 0$$
$$D = 169 - (4 * (-2) * 7) = 169 + (14 * 4) = 169 + 56 = 225$$
$$x_{12} = \frac{-13 \pm 15}{-4} = [-0.5, 7$$

Ответ: \textbf{-0.5}

\section{№ \textbf{77372}}

$$\frac{x + 8}{5x + 7} = \frac{x + 8}{7x + 5}$$
$$\frac{(x+8)(7x+5)}{(5x + 7)(7x + 5)} = \frac{(x+8)(5x + 7)}{(5x + 7)(7x + 5)}$$
$$\frac{7x^2 + 5x + 56x + 40}{(5x+7)(7x+5)} = \frac{5x^2 + 7x + 40x + 56}{(5x+7)(7x+5)}$$
$$7x^2 + 5x + 56x + 40 = 5x^2 + 7x + 40x + 56$$
$$ 7x^2 - 5x^2 + 61x - 47x + 40 - 56 = 0 $$
$$ 2x^2 + 14x - 16 = 0 $$
$$ D = 196 - (4 * 2 * (-16)) = 196 + (8 * 16) = 196 + 128 = 324 $$
$$ x_{12} = \frac{-14 \pm 18}{4} = [1, -8$$

ОДЗ:

$$5x + 7 \neq 0 $$
$$5x \neq -7$$
$$x \neq -\frac{7}{5}$$


$$7x + 5 \neq 0 $$
$$7x \neq -5 $$
$$x \neq -\frac{5}{7}$$

Ответ: \textbf{1}
\end{document}
