% !TEX TS-program = pdflatex
% !TEX encoding = UTF-8 Unicode

% This is a simple template for a LaTeX document using the "article" class.
% See "book", "report", "letter" for other types of document.

\documentclass[11pt]{article} % use larger type; default would be 10pt

\usepackage[utf8]{inputenc} % set input encoding (not needed with XeLaTeX)
\usepackage[english,russian]{babel}

\usepackage{amsmath}

%%% Examples of Article customizations
% These packages are optional, depending whether you want the features they provide.
% See the LaTeX Companion or other references for full information.

%%% PAGE DIMENSIONS
\usepackage{geometry} % to change the page dimensions
\geometry{a4paper} % or letterpaper (US) or a5paper or....
% \geometry{margin=2in} % for example, change the margins to 2 inches all round
% \geometry{landscape} % set up the page for landscape
%   read geometry.pdf for detailed page layout information

\usepackage{graphicx} % support the \includegraphics command and options

% \usepackage[parfill]{parskip} % Activate to begin paragraphs with an empty line rather than an indent

%%% PACKAGES
\usepackage{booktabs} % for much better looking tables
\usepackage{array} % for better arrays (eg matrices) in maths
\usepackage{paralist} % very flexible & customisable lists (eg. enumerate/itemize, etc.)
\usepackage{verbatim} % adds environment for commenting out blocks of text & for better verbatim
\usepackage{subfig} % make it possible to include more than one captioned figure/table in a single float
% These packages are all incorporated in the memoir class to one degree or another...

%%% HEADERS & FOOTERS
\usepackage{fancyhdr} % This should be set AFTER setting up the page geometry
\pagestyle{fancy} % options: empty , plain , fancy
\renewcommand{\headrulewidth}{0pt} % customise the layout...
\lhead{}\chead{}\rhead{}
\lfoot{}\cfoot{\thepage}\rfoot{}

%%% SECTION TITLE APPEARANCE
\usepackage{sectsty}
\allsectionsfont{\sffamily\mdseries\upshape} % (See the fntguide.pdf for font help)
% (This matches ConTeXt defaults)

%%% ToC (table of contents) APPEARANCE
\usepackage[nottoc,notlof,notlot]{tocbibind} % Put the bibliography in the ToC
\usepackage[titles,subfigure]{tocloft} % Alter the style of the Table of Contents
\renewcommand{\cftsecfont}{\rmfamily\mdseries\upshape}
\renewcommand{\cftsecpagefont}{\rmfamily\mdseries\upshape} % No bold!

%%% END Article customizations

%%% The "real" document content comes below...

\title{Линейные, квадратные, кубические уравнения}
\author{Lisid Lakonsky}
%\date{} % Activate to display a given date or no date (if empty),
         % otherwise the current date is printed 

\begin{document}
\maketitle

\section{№ \textbf{26662}}

$$ \frac{4}{7}x = 7\frac{3}{7} $$
$$ \frac{4}{7}x = \frac{52}{7} $$
$$ x = \frac{52 * 7}{7 * 4} $$
$$ x = \frac{52}{4} = 13 $$

Ответ: \textbf{13}

\section{№ \textbf{26663}}

$$ -\frac{2}{9}x = 1\frac{1}{9} $$
$$ -\frac{2}{9}x = \frac{10}{9} $$
$$ -2x = 10 $$
$$ x = -5 $$

Ответ: \textbf{-5}

\section{№ \textbf{510118}}

$$ (x - 10)^2 = (x+4)^2 $$
$$ (x-10)^2 - (x+4)^2 = 0$$
$$ ((x - 10) - (x + 4))((x-10)+(x+4)) = 0$$
$$ -14 * (2x + 6) = 0 $$
$$ -28x + 84 = 0 $$
$$ -28x = -84 $$
$$ x = 3 $$

Ответ: \textbf{3}

\section{№ \textbf{77368}}

$$ (2x + 7)^2 = (2x - 1)^2 $$
$$ (2x + 7)^2 - (2x - 1)^2 = 0 $$
$$ (2x+7-2x+1)(2x+7+2x-1) = 0 $$
$$ 8 * (4x + 6) = 0 $$
$$ 32x + 48 = 0 $$
$$ 32x = -48 $$
$$ 8x = -12 $$
$$ x = -\frac{12}{8} = -1.5 $$

Ответ: \textbf{-1.5}

\section{№ \textbf{77369}}

$$ (x-6)^2 = -24x $$
$$ x^2 - 12x + 36 = -24x $$
$$ x^2 + 12x + 36 = 0 $$
$$ D = 12^2 - 4 * 1 * 36 = 144 - 144 = 0 $$
$$ x = -\frac{12}{2} = -6 $$

Ответ: \textbf{-6}

\section{№ \textbf{282850}}

$$ (x-1)^3 = -8 $$
$$ \sqrt[3]{(x - 1)^3} = \sqrt[3]{-8} $$
$$ x - 1 = -2 $$
$$ x = -1 $$

Ответ: \textbf{-1}
 
\end{document}
