% !TEX TS-program = pdflatex
% !TEX encoding = UTF-8 Unicode

% This is a simple template for a LaTeX document using the "article" class.
% See "book", "report", "letter" for other types of document.

\documentclass[11pt]{article} % use larger type; default would be 10pt

\usepackage[utf8]{inputenc} % set input encoding (not needed with XeLaTeX)
\usepackage[english,russian]{babel}

\usepackage{amsmath}

%%% Examples of Article customizations
% These packages are optional, depending whether you want the features they provide.
% See the LaTeX Companion or other references for full information.

%%% PAGE DIMENSIONS
\usepackage{geometry} % to change the page dimensions
\geometry{a4paper} % or letterpaper (US) or a5paper or....
% \geometry{margin=2in} % for example, change the margins to 2 inches all round
% \geometry{landscape} % set up the page for landscape
%   read geometry.pdf for detailed page layout information

\usepackage{graphicx} % support the \includegraphics command and options

% \usepackage[parfill]{parskip} % Activate to begin paragraphs with an empty line rather than an indent

%%% PACKAGES
\usepackage{booktabs} % for much better looking tables
\usepackage{array} % for better arrays (eg matrices) in maths
\usepackage{paralist} % very flexible & customisable lists (eg. enumerate/itemize, etc.)
\usepackage{verbatim} % adds environment for commenting out blocks of text & for better verbatim
\usepackage{subfig} % make it possible to include more than one captioned figure/table in a single float
% These packages are all incorporated in the memoir class to one degree or another...

%%% HEADERS & FOOTERS
\usepackage{fancyhdr} % This should be set AFTER setting up the page geometry
\pagestyle{fancy} % options: empty , plain , fancy
\renewcommand{\headrulewidth}{0pt} % customise the layout...
\lhead{}\chead{}\rhead{}
\lfoot{}\cfoot{\thepage}\rfoot{}

%%% SECTION TITLE APPEARANCE
\usepackage{sectsty}
\allsectionsfont{\sffamily\mdseries\upshape} % (See the fntguide.pdf for font help)
% (This matches ConTeXt defaults)

%%% ToC (table of contents) APPEARANCE
\usepackage[nottoc,notlof,notlot]{tocbibind} % Put the bibliography in the ToC
\usepackage[titles,subfigure]{tocloft} % Alter the style of the Table of Contents
\renewcommand{\cftsecfont}{\rmfamily\mdseries\upshape}
\renewcommand{\cftsecpagefont}{\rmfamily\mdseries\upshape} % No bold!

%%% END Article customizations

%%% The "real" document content comes below...

\title{Классическое определение вероятности}
\author{Lisid Lakonsky}
%\date{} % Activate to display a given date or no date (if empty),
         % otherwise the current date is printed 

\begin{document}
\maketitle

\section{№ \textbf{325913}}

$$P_n = n!$$
$$P_9 = 9!$$
$$P = 1 - \frac{k}{P_9}$$
$$k = 9 * 2 * 1 * 7!$$
$$P = 1 - \frac{9 * 2 * 1 * 7!}{9!} = 1 - \frac{1 / 4} = 0.75$$

Ответ: \textbf{0,75}

\section{№ \textbf{320181}}

P = \frac{2}{5} = 0.4

Ответ: \textbf{0.4}

\section{№ \textbf{509110}}

Всего в копилке лежит: $7 + 5 * 2 + 6 * 5 + 2 * 10 = 7 + 10 + 30 + 20 = 67$

Оставшаяся в копилке сумма составит менее 60 рублей в следующих случаях:

\begin{enumerate}
	\item Если Дина вытащит десятирублевую монету
\end{enumerate}

Всего монет $20$

Таким образом,

$$P = \frac{2}{20} = 0.1$$

Ответ: \textbf{0.1}

\section{№ \textbf{320183}}

Возможные исходы:

\begin{enumerate}
	\item101
	\item 011
	\item 110
\end{enumerate}

Всего исходов: $2^3$

$$P(A) = \frac{3}{2^3} = \frac{3}{8} = 0.375$

Ответ: \textbf{0.375}

\section{№ \textbf{283467}}

$$P(A) = \frac{3}{2^3} = 0.375$

Ответ: \textbf{0,375}

\section{№ \textbf{282853}}

Благоприятные исходы: 

\begin{enumerate}
	\item 2 + 6
	\item 6 + 2
	\item 3 + 5
	\item 5 + 3
	\item 4 + 4
\end{enumerate}

Всего исходов: $6^2 = 36$

$$P = \frac{5}{36} = 0,1388$

Ответ: \textbf{0,14}

\section{№ \textbf{282858}}

$$P = \frac{9}{25} = 0.36$

Ответ: \textbf{0.36}

\section{№ \textbf{285924}}

$$P = \frac{3}{10} = 0.3$$

Ответ: \textbf{0,3}

\section{№ \textbf{320189}}

$x = \frac{5000 - 2512}{5000} = 0.498$

Ответ: \textbf{0,498}

\end{document}
