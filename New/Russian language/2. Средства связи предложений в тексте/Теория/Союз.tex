% !TEX TS-program = pdflatex
% !TEX encoding = UTF-8 Unicode

% This is a simple template for a LaTeX document using the "article" class.
% See "book", "report", "letter" for other types of document.

\documentclass[11pt]{article} % use larger type; default would be 10pt

\usepackage[utf8]{inputenc} % set input encoding (not needed with XeLaTeX)
\usepackage[english,russian]{babel}

%%% Examples of Article customizations
% These packages are optional, depending whether you want the features they provide.
% See the LaTeX Companion or other references for full information.

%%% PAGE DIMENSIONS
\usepackage{geometry} % to change the page dimensions
\geometry{a4paper} % or letterpaper (US) or a5paper or....
% \geometry{margin=2in} % for example, change the margins to 2 inches all round
% \geometry{landscape} % set up the page for landscape
%   read geometry.pdf for detailed page layout information

\usepackage{graphicx} % support the \includegraphics command and options

% \usepackage[parfill]{parskip} % Activate to begin paragraphs with an empty line rather than an indent

%%% PACKAGES
\usepackage{booktabs} % for much better looking tables
\usepackage{array} % for better arrays (eg matrices) in maths
\usepackage{paralist} % very flexible & customisable lists (eg. enumerate/itemize, etc.)
\usepackage{verbatim} % adds environment for commenting out blocks of text & for better verbatim
\usepackage{subfig} % make it possible to include more than one captioned figure/table in a single float
% These packages are all incorporated in the memoir class to one degree or another...

%%% HEADERS & FOOTERS
\usepackage{fancyhdr} % This should be set AFTER setting up the page geometry
\pagestyle{fancy} % options: empty , plain , fancy
\renewcommand{\headrulewidth}{0pt} % customise the layout...
\lhead{}\chead{}\rhead{}
\lfoot{}\cfoot{\thepage}\rfoot{}

%%% SECTION TITLE APPEARANCE
\usepackage{sectsty}
\allsectionsfont{\sffamily\mdseries\upshape} % (See the fntguide.pdf for font help)
% (This matches ConTeXt defaults)

%%% ToC (table of contents) APPEARANCE
\usepackage[nottoc,notlof,notlot]{tocbibind} % Put the bibliography in the ToC
\usepackage[titles,subfigure]{tocloft} % Alter the style of the Table of Contents
\renewcommand{\cftsecfont}{\rmfamily\mdseries\upshape}
\renewcommand{\cftsecpagefont}{\rmfamily\mdseries\upshape} % No bold!

%%% END Article customizations

%%% The "real" document content comes below...

\title{Теория. Союз}
\author{Lisid Lakonsky}
%\date{} % Activate to display a given date or no date (if empty),
         % otherwise the current date is printed 

\begin{document}
\maketitle

\textbf{Союз - это служебная часть речи, связывающая однородные члены, простые предложения в составе сложного, а также предложения в тексте.}


По \emph{строению} союзы делятся на \emph{простые} и \emph{составные}. По \emph{значению} союзы делятся на \emph{сочинительные} и \emph{подчинительные}.


\textbf{Сочинительные союзы - это союзы, которые служат для связи однородных членов предложения и равноправных по смыслу простых предложений в составе сложного.}

\vspace{5mm}

\begin{tabular}{ | p{100pt} | p{100pt} | p{200pt} | }
\hline
\textbf{Виды сочинительных союзов} & \textbf{Значение} & \textbf{Примеры} \\ \hline
\textbf{Соединительные} & Соединительные союзы используются для выражения одновременно или последовательно происходящих событий, явлений действительности & и, да(=и), ни-ни, тоже, также \\ \hline
\textbf{Противительные} & Противительные союзы выражают отношения противопоставления или разграничения & а, но, да(=но), зато, однако, же \\ \hline
\textbf{Разделительные} & Разделительные союзы вносят в предложение значения чередования, выбора, предложения, неразличения & или, либо, то-то, то ли ... то ли, не то ... не то \\
\hline
\end{tabular}

\vspace{5mm}

\textbf{Подчинительные союзы - это союзы, которые связывают простые предложения в составе сложноподчиненного предложения. В таком сложном предложении от одного предложения к другому можно поставить вопрос.}

\begin{tabular}{ | p{100pt} | p{300pt} | }
\hline
\textbf{Виды изъяснительных союзов} & \textbf{Примеры} \\ \hline
\textbf{Изъяснительные} & как, чтобы, что, будто \\ \hline
\textbf{Временные} & когда, как, как только, между тем как, лишь, лишь только, едва лишь, пока \\ \hline
\textbf{Причинные} & ибо, потому что, оттого что, так как, из-за того что, благодаря тому что, вследствие того что, в связи с тем что \\ \hline
\textbf{Целевые} & чтобы (чтоб), дабы, для того чтобы, с тем чтобы \\ \hline
\textbf{Условные} & если, если бы, ежели, ежели бы, коли (коль), когда, когда бы, раз \\ \hline
\textbf{Уступительные} & хотя (хоть), хотя бы, пусть, даром что, несмотря на то что, невзирая на то что \\ \hline
\textbf{Сравнительные} & как, как бы, как будто, будто, будто бы, словно, словно как, точно \\ \hline
\textbf{Следствия} & так что \\ \hline
\end{tabular}

\end{document}
