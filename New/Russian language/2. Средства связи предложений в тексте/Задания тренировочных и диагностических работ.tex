% !TEX TS-program = pdflatex
% !TEX encoding = UTF-8 Unicode

% This is a simple template for a LaTeX document using the "article" class.
% See "book", "report", "letter" for other types of document.

\documentclass[11pt]{article} % use larger type; default would be 10pt

\usepackage[utf8]{inputenc} % set input encoding (not needed with XeLaTeX)
\usepackage[english,russian]{babel}

\usepackage{amsmath}

%%% Examples of Article customizations
% These packages are optional, depending whether you want the features they provide.
% See the LaTeX Companion or other references for full information.

%%% PAGE DIMENSIONS
\usepackage{geometry} % to change the page dimensions
\geometry{a4paper} % or letterpaper (US) or a5paper or....
% \geometry{margin=2in} % for example, change the margins to 2 inches all round
% \geometry{landscape} % set up the page for landscape
%   read geometry.pdf for detailed page layout information

\usepackage{graphicx} % support the \includegraphics command and options

% \usepackage[parfill]{parskip} % Activate to begin paragraphs with an empty line rather than an indent

%%% PACKAGES
\usepackage{booktabs} % for much better looking tables
\usepackage{array} % for better arrays (eg matrices) in maths
\usepackage{paralist} % very flexible & customisable lists (eg. enumerate/itemize, etc.)
\usepackage{verbatim} % adds environment for commenting out blocks of text & for better verbatim
\usepackage{subfig} % make it possible to include more than one captioned figure/table in a single float
% These packages are all incorporated in the memoir class to one degree or another...

%%% HEADERS & FOOTERS
\usepackage{fancyhdr} % This should be set AFTER setting up the page geometry
\pagestyle{fancy} % options: empty , plain , fancy
\renewcommand{\headrulewidth}{0pt} % customise the layout...
\lhead{}\chead{}\rhead{}
\lfoot{}\cfoot{\thepage}\rfoot{}

%%% SECTION TITLE APPEARANCE
\usepackage{sectsty}
\allsectionsfont{\sffamily\mdseries\upshape} % (See the fntguide.pdf for font help)
% (This matches ConTeXt defaults)

%%% ToC (table of contents) APPEARANCE
\usepackage[nottoc,notlof,notlot]{tocbibind} % Put the bibliography in the ToC
\usepackage[titles,subfigure]{tocloft} % Alter the style of the Table of Contents
\renewcommand{\cftsecfont}{\rmfamily\mdseries\upshape}
\renewcommand{\cftsecpagefont}{\rmfamily\mdseries\upshape} % No bold!

%%% END Article customizations

%%% The "real" document content comes below...

\title{Задания тренировочных и диагностических работ}
\author{Lisid Lakonsky}
%\date{} % Activate to display a given date or no date (if empty),
         % otherwise the current date is printed 

\begin{document}
\maketitle

\section{№ \textbf{14427}}

Ответ: ---

Правильный ответ: \textbf{вслучаеесли}

\section{№ \textbf{14413}}

Ответ: \textbf{такимобразом}

\section{№ \textbf{14340}}

Ответ: или -- дурак
Правильный ответ: \textbf{атакже}

\section{№ \textbf{14415}}

Ответ: наконец -- это не частица, а наречие
Правильный ответ: \textbf{все-таки}

\section{№ \textbf{14332}}

Ответ: \textbf{напротив}

\section{№ \textbf{14344}}

Ответ: \textbf{но}

\section{№ \textbf{16968}}

Ответ: очень -- дурак
Правильный ответ: \textbf{настолько}

\section{№ \textbf{33918}}

Ответ: \textbf{такчто}

\section{№ \textbf{37254}}

Ответ: \textbf{перед}

\section{№ \textbf{14335}}

Ответ: \textbf{так}

\section{№ \textbf{14355}}

Ответ: ---

Правильный ответ: \textbf{деловтомчто}

\section{№ \textbf{14362}}

Ответ: \textbf{ктомуже}

\section{№ \textbf{14328}}

Ответ: следовательно -- дурак
Правильный ответ: \textbf{безусловно}

\section{№ \textbf{14360}}

Ответ: ---

Правильный ответ: \textbf{неслучайно}

\section{№ \textbf{14341}}

Ответ: \textbf{согласно}

\section{№ \textbf{14417}}

Ответ: \textbf{такимобразом}

\section{№ \textbf{14345}}

Ответ: \textbf{ведь}

\section{№ \textbf{16857}}

Ответ: \textbf{например}

\section{№ \textbf{14343}}

Ответ: \textbf{такимобразом}

\section{№ \textbf{14412}}

Ответ: то -- неправильно. это местоимение, а надо наречие

Правильный ответ: \textbf{поэтому}

\section{№ \textbf{14419}}

Ответ: \textbf{каклюбое}

\section{№ \textbf{35439}}

Ответ: эта -- не до конца прочитал и ошибся

Правильный ответ: \textbf{иная}

\section{№ \textbf{14433}}

Ответ: \textbf{лишь}

\section{№ \textbf{14429}}

Ответ: \textbf{оказалось}

\section{№ \textbf{14329}}

Ответ: \textbf{так}

\section{№ \textbf{14436}}

Ответ: \textbf{когда}

\section{№ \textbf{29834}}

Ответ: \textbf{или}

\section{№ \textbf{14337}}

Ответ: \textbf{даже}

\section{№ \textbf{14428}}

Ответ: \textbf{когда}

\section{№ \textbf{14430}}

Ответ: \textbf{именно}

\section{№ \textbf{14411}}

Ответ: \textbf{поэтому}

\section{№ \textbf{14861}}

Ответ: \textbf{но}

\section{№ \textbf{14352}}

Ответ: \textbf{именно}

\section{№ \textbf{14888}}

Ответ: \textbf{ибо}

\section{№ \textbf{14339}}

Ответ: \textbf{такимобразом}

\section{№ \textbf{14431}}

Ответ: \textbf{такимобразом}

\section{№ \textbf{18173}}

Ответ: \textbf{таккак}

\section{№ \textbf{14416}}

Ответ: \textbf{поэтому}

\section{№ \textbf{14322}}

Ответ: \textbf{поэтому}

\end{document}
