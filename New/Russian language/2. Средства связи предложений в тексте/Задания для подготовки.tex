% !TEX TS-program = pdflatex
% !TEX encoding = UTF-8 Unicode

% This is a simple template for a LaTeX document using the "article" class.
% See "book", "report", "letter" for other types of document.

\documentclass[11pt]{article} % use larger type; default would be 10pt

\usepackage[utf8]{inputenc} % set input encoding (not needed with XeLaTeX)
\usepackage[english,russian]{babel}

%%% Examples of Article customizations
% These packages are optional, depending whether you want the features they provide.
% See the LaTeX Companion or other references for full information.

%%% PAGE DIMENSIONS
\usepackage{geometry} % to change the page dimensions
\geometry{a4paper} % or letterpaper (US) or a5paper or....
% \geometry{margin=2in} % for example, change the margins to 2 inches all round
% \geometry{landscape} % set up the page for landscape
%   read geometry.pdf for detailed page layout information

\usepackage{graphicx} % support the \includegraphics command and options

% \usepackage[parfill]{parskip} % Activate to begin paragraphs with an empty line rather than an indent

%%% PACKAGES
\usepackage{booktabs} % for much better looking tables
\usepackage{array} % for better arrays (eg matrices) in maths
\usepackage{paralist} % very flexible & customisable lists (eg. enumerate/itemize, etc.)
\usepackage{verbatim} % adds environment for commenting out blocks of text & for better verbatim
\usepackage{subfig} % make it possible to include more than one captioned figure/table in a single float
% These packages are all incorporated in the memoir class to one degree or another...

%%% HEADERS & FOOTERS
\usepackage{fancyhdr} % This should be set AFTER setting up the page geometry
\pagestyle{fancy} % options: empty , plain , fancy
\renewcommand{\headrulewidth}{0pt} % customise the layout...
\lhead{}\chead{}\rhead{}
\lfoot{}\cfoot{\thepage}\rfoot{}

%%% SECTION TITLE APPEARANCE
\usepackage{sectsty}
\allsectionsfont{\sffamily\mdseries\upshape} % (See the fntguide.pdf for font help)
% (This matches ConTeXt defaults)

%%% ToC (table of contents) APPEARANCE
\usepackage[nottoc,notlof,notlot]{tocbibind} % Put the bibliography in the ToC
\usepackage[titles,subfigure]{tocloft} % Alter the style of the Table of Contents
\renewcommand{\cftsecfont}{\rmfamily\mdseries\upshape}
\renewcommand{\cftsecpagefont}{\rmfamily\mdseries\upshape} % No bold!

%%% END Article customizations

%%% The "real" document content comes below...

\title{Задания для подготовки}
\author{Lisid Lakonsky}
%\date{} % Activate to display a given date or no date (if empty),
         % otherwise the current date is printed 

\begin{document}
\maketitle

\section{№ \textbf{1262}}

\paragraph{Ответ: такимобразом -- ошибка. Было сказано подобрать вводное слово}

\paragraph{Правильный ответ: \textbf{так}}

\section{№ \textbf{1300}}

\paragraph{Ответ: \textbf{только}}

\section{№ \textbf{1338}}

\paragraph{Ответ: для этого -- ошибка.}

\paragraph{Правильный ответ: \textbf{именно поэтому}}

\section{№ \textbf{1376}}

\paragraph{Ответ: \textbf{но}}

\section{№ \textbf{1414}}

\paragraph{Ответ: ----}

\paragraph{Правильный ответ: \textbf{оказалось}}

\section{№ \textbf{1452}}

\paragraph{Ответ: \textbf{но}}

\section{№ \textbf{1490}}

\paragraph{Ответ: \textbf{эти}}

\section{№ \textbf{1528}}

\paragraph{Ответ: \textbf{так}}

\section{№ \textbf{1566}}

\paragraph{Ответ: \textbf{поэтому}}

\section{№ \textbf{1604}}

\paragraph{Ответ: \textbf{поэтому}}

\section{№ \textbf{1642}}

\paragraph{Ответ: например -- ошибка. Просили вводное словосочетание}

\paragraph{Правильный ответ: \textbf{таким образом}}

\section{№ \textbf{1680}}

\paragraph{Ответ: \textbf{это}}

\section{№ \textbf{2915}}

\paragraph{Ответ: \textbf{лишь}}

\section{№ \textbf{14056}}

\paragraph{Ответ: \textbf{ибо}}

\section{№ \textbf{14246}}

\paragraph{Ответ: \textbf{вследствие этого}}

\section{№ \textbf{14247}}

\paragraph{Ответ: \textbf{этим}}

\section{№ \textbf{14248}}

\paragraph{Ответ: \textbf{эти}}

\section{№ \textbf{14249}}

\paragraph{Ответ: \textbf{в отличие от}}

\section{№ \textbf{14250}}

\paragraph{Ответ: однако -- ошибка}

\paragraph{Правильный ответ: \textbf{теперь}}

\section{№ \textbf{14251}}

\paragraph{Ответ: однако -- ошибка}

\paragraph{Правильный ответ: \textbf{с другой стороны}}

\section{№ \textbf{14252}}

\paragraph{Ответ: \textbf{но}}

\section{№ \textbf{14253}}

\paragraph{Ответ: \textbf{несмотря на}}

\section{№ \textbf{14254}}

\paragraph{Ответ: \textbf{такой}}

\end{document}
