% !TEX TS-program = pdflatex
% !TEX encoding = UTF-8 Unicode

% This is a simple template for a LaTeX document using the "article" class.
% See "book", "report", "letter" for other types of document.

\documentclass[11pt]{article} % use larger type; default would be 10pt

\usepackage[utf8]{inputenc} % set input encoding (not needed with XeLaTeX)
\usepackage[english,russian]{babel}

\usepackage{amsmath}

%%% Examples of Article customizations
% These packages are optional, depending whether you want the features they provide.
% See the LaTeX Companion or other references for full information.

%%% PAGE DIMENSIONS
\usepackage{geometry} % to change the page dimensions
\geometry{a4paper} % or letterpaper (US) or a5paper or....
% \geometry{margin=2in} % for example, change the margins to 2 inches all round
% \geometry{landscape} % set up the page for landscape
%   read geometry.pdf for detailed page layout information

\usepackage{graphicx} % support the \includegraphics command and options

% \usepackage[parfill]{parskip} % Activate to begin paragraphs with an empty line rather than an indent

%%% PACKAGES
\usepackage{booktabs} % for much better looking tables
\usepackage{array} % for better arrays (eg matrices) in maths
\usepackage{paralist} % very flexible & customisable lists (eg. enumerate/itemize, etc.)
\usepackage{verbatim} % adds environment for commenting out blocks of text & for better verbatim
\usepackage{subfig} % make it possible to include more than one captioned figure/table in a single float
% These packages are all incorporated in the memoir class to one degree or another...

%%% HEADERS & FOOTERS
\usepackage{fancyhdr} % This should be set AFTER setting up the page geometry
\pagestyle{fancy} % options: empty , plain , fancy
\renewcommand{\headrulewidth}{0pt} % customise the layout...
\lhead{}\chead{}\rhead{}
\lfoot{}\cfoot{\thepage}\rfoot{}

%%% SECTION TITLE APPEARANCE
\usepackage{sectsty}
\allsectionsfont{\sffamily\mdseries\upshape} % (See the fntguide.pdf for font help)
% (This matches ConTeXt defaults)

%%% ToC (table of contents) APPEARANCE
\usepackage[nottoc,notlof,notlot]{tocbibind} % Put the bibliography in the ToC
\usepackage[titles,subfigure]{tocloft} % Alter the style of the Table of Contents
\renewcommand{\cftsecfont}{\rmfamily\mdseries\upshape}
\renewcommand{\cftsecpagefont}{\rmfamily\mdseries\upshape} % No bold!

%%% END Article customizations

%%% The "real" document content comes below...

\title{Монотонные функции}
\author{Lisid Lakonsky}
%\date{} % Activate to display a given date or no date (if empty),
         % otherwise the current date is printed 

\begin{document}
\maketitle

\section{№ \textbf{13478}}

Выражение: $(x \land z) \lor (x \land \lnot{y} \land \lnot{z})$

\vspace{2mm}
\begin{tabular}{ | l | l | l | l | }
\hline
Перем. 1 & Перем. 2 & Перем. 3 & Функция \\ \hline
??? & ??? & ??? & F \\ \hline
0 & 1 & 0 & 1 \\ \hline
0 & 1 & 1 & 1 \\ \hline
1 & 1 & 1 & 1 \\ \hline
\end{tabular}

\vspace{2mm}

Результат функции равен единице в следующих случаях:

\begin{enumerate}
	\item x = 1 и z = 1
	\item x = 1 и y = 0 и z = 0
\end{enumerate}

Таким образом, мы видим, рассматривается первый случай, что x - это вторая переменная.

Рассматривая второй случай, мы видим, что z - третья переменная. Таким образом, оставшаяся переменная - y

Ответ: \textbf{yxz}

\section{№ \textbf{13505}}

Выражение: $(x \land \lnot{z}) \lor (x \land y \land z)$

\vspace{2mm}
\begin{tabular}{ | l | l | l | l | }
\hline
Перем. 1 & Перем. 2 & Перем. 3 & Функция \\ \hline
??? & ??? & ??? & F \\ \hline
0 & 0 & 1 & 1 \\ \hline
1 & 0 & 1 & 1 \\ \hline
1 & 1 & 1 & 1 \\ \hline
\end{tabular}

\vspace{2mm}

Результат функции равен единице в следующих случаях:

\begin{enumerate}
	\item x = 1 и z = 0
	\item x = 1, y = 1, z = 1
\end{enumerate}

Рассматривая первый случай, мы видим, что третья переменная - это икс.

Рассматривая второй случай, мы видим, что вторая переменная - это зет.

Рассматривая третий случай, мы понимаем, что оставшаяся переменная, первая - это игрек.

Ответ: \textbf{yzx}

\section{№ \textbf{13586}}

Выражение: $\lnot{w} \land z \land (y \rightarrow x)$

\vspace{2mm}
\begin{tabular}{ | l | l | l | l | l | }
\hline
Перем. 1 & Перем. 2 & Перем. 3 & Перем 4. Функция \\ \hline
??? & ??? & ??? & ??? & F \\ \hline
1 & 0 & 0 & 0 & 1 \\ \hline
1 & 0 & 1 & 0 & 1 \\ \hline
1 & 0 & 1 & 1 & 1 \\ \hline
\end{tabular}

\vspace{2mm}

Результат функции равен единице в следующих случаях:

\begin{enumerate}
	\item w = 0, z = 1, y = 1, x = 1
	\item w = 0, z = 1, y = 0, x = 1
	\item w = 0, z = 1, y = 0, x = 0
\end{enumerate}

Рассматривая первый случай, мы понимаем, что первая переменная - это зет, ибо только она всегда равна единице

Рассматривая второй случай, мы понимаем, что третья переменная - это иск.

Рассматривая третий случай, мы понимаем, что вторая переменная - это w, а оставшаяся, четвертая, это игрек.

Ответ: \textbf{zwxy}

\section{№ \textbf{10278}}

Выражение: $(\lnot{x} \land z) \lor (\lnot{x} \land \lnot{y} \land \lnot{z})$

\vspace{2mm}
\begin{tabular}{ | l | l | l | l | }
\hline
Перем. 1 & Перем. 2 & Перем. 3 & Функция \\ \hline
??? & ??? & ??? & F \\ \hline
0 & 0 & 0 & 1 \\ \hline
0 & 0 & 1 & 1 \\ \hline
0 & 1 & 1 & 1 \\ \hline
\end{tabular}

\vspace{2mm}

Результат функции равен единице в следующих случаях:

\begin{enumerate}
	\item x = 0, z = 1
	\item x = 0, y = 0, z = 0
\end{enumerate}

Рассматривая первый случай, мы не понимаем ничего.

Рассматривая второй случай, мы понимаем, что третья переменная - это зет

Рассматривая третий случай, мы понимаем, что первая переменная - это икс.

А оставшаяся, вторая переменная, получается, является игреком.

Ответ: \textbf{xyz}

\section{№ \textbf{13613}}

Выражение: $\lnot{x} \land y \land (z \rightarrow w)$

\vspace{2mm}
\begin{tabular}{ | l | l | l | l | l | }
\hline
Перем. 1 & Перем. 2 & Перем. 3 & Перем 4. Функция \\ \hline
??? & ??? & ??? & ??? & F \\ \hline
1 & 0 & 0 & 0 & 1 \\ \hline
1 & 0 & 1 & 0 & 1 \\ \hline
1 & 0 & 1 & 1 & 1 \\ \hline
\end{tabular}

\vspace{2mm}

Результат функции равен единице в следующих случаях:

\begin{enumerate}
	\item x = 0, y = 1, z = 0, w = 1
	\item x = 0, y = 1, z = 0, w = 0
	\item x = 0, y = 1, z = 1, w = 1
\end{enumerate}

Рассматривая всю таблицу целиком, мы понимаем, что первая переменная - это игрек, вторая - иск, третья - w, четвертая - зет.

Ответ: \textbf{yxwz}

\end{document}
