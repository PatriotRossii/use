% !TEX TS-program = pdflatex
% !TEX encoding = UTF-8 Unicode

% This is a simple template for a LaTeX document using the "article" class.
% See "book", "report", "letter" for other types of document.

\documentclass[11pt]{article} % use larger type; default would be 10pt

\usepackage[utf8]{inputenc} % set input encoding (not needed with XeLaTeX)
\usepackage[english,russian]{babel}

%%% Examples of Article customizations
% These packages are optional, depending whether you want the features they provide.
% See the LaTeX Companion or other references for full information.

%%% PAGE DIMENSIONS
\usepackage{geometry} % to change the page dimensions
\geometry{a4paper} % or letterpaper (US) or a5paper or....
% \geometry{margin=2in} % for example, change the margins to 2 inches all round
% \geometry{landscape} % set up the page for landscape
%   read geometry.pdf for detailed page layout information

\usepackage{graphicx} % support the \includegraphics command and options

% \usepackage[parfill]{parskip} % Activate to begin paragraphs with an empty line rather than an indent

%%% PACKAGES
\usepackage{booktabs} % for much better looking tables
\usepackage{array} % for better arrays (eg matrices) in maths
\usepackage{paralist} % very flexible & customisable lists (eg. enumerate/itemize, etc.)
\usepackage{verbatim} % adds environment for commenting out blocks of text & for better verbatim
\usepackage{subfig} % make it possible to include more than one captioned figure/table in a single float
% These packages are all incorporated in the memoir class to one degree or another...

%%% HEADERS & FOOTERS
\usepackage{fancyhdr} % This should be set AFTER setting up the page geometry
\pagestyle{fancy} % options: empty , plain , fancy
\renewcommand{\headrulewidth}{0pt} % customise the layout...
\lhead{}\chead{}\rhead{}
\lfoot{}\cfoot{\thepage}\rfoot{}

%%% SECTION TITLE APPEARANCE
\usepackage{sectsty}
\allsectionsfont{\sffamily\mdseries\upshape} % (See the fntguide.pdf for font help)
% (This matches ConTeXt defaults)

%%% ToC (table of contents) APPEARANCE
\usepackage[nottoc,notlof,notlot]{tocbibind} % Put the bibliography in the ToC
\usepackage[titles,subfigure]{tocloft} % Alter the style of the Table of Contents
\renewcommand{\cftsecfont}{\rmfamily\mdseries\upshape}
\renewcommand{\cftsecpagefont}{\rmfamily\mdseries\upshape} % No bold!

%%% END Article customizations

%%% The "real" document content comes below...

\title{Поиск оптимального маршрута по таблице}
\author{Lisid Lakonsky}
%\date{} % Activate to display a given date or no date (if empty),
         % otherwise the current date is printed 

\begin{document}
\maketitle

\section{№ \textbf{6794}}

\begin{enumerate}
	\item A - B - C - D - E - Z = 4 + 3 + 13 + 4 + 8 = 32
	\item A - B - C - D - F - Z = 4 + 3 + 13 + 7 + 2 = 29
	\item A - B - C - D - Z = 4 + 3 + 13 + 10 = 30
\end{enumerate}

Ответ: \textbf{29}

\section{№ \textbf{6794}}

\begin{enumerate}
	\item A - B - C - D - E - F - Z = 8 + 5 + 3 + 2 + 2 + 3 = 23
	\item A - B - C - D - E - Z = 8 + 5 + 3 + 2 + 6 = 24
	\item A - B - D - E - F - Z = 8 + 7 + 2 + 2 + 3 = 22
\end{enumerate}

Ответ: \textbf{22}

\section{№ \textbf{6321}}

\begin{enumerate}
	\item A - B - C - D - E - F = 2 + 4 + 3 + 3 + 2 = 14
	\item A - B - C - D -  F = 2 + 4 + 3 + 6 = 15
	\item A - B - D - E - F = 2 + 8 + 3 + 2 = 15
	\item A - B - D - F = 2 + 8 + 6 = 16
\end{enumerate}

Ответ: \textbf{14}

\section{№ \textbf{3487}}

\begin{enumerate}
	\item A - D - C - B = 1 + 2 + 4 = 7
	\item A - D - F - B = 1 + 2 + 5 = 8
	\item A - F - B = 3 + 5 = 8
	\item A - F - D - C - B = 3 + 2 + 2 + 4 = 11
\end{enumerate}

Ответ: \textbf{7}

\section{№ \textbf{7323}}

\begin{enumerate}
	\item A - B - D - Z = 5 + 21 + 12 = 38
	\item A - C - D - E - Z = 7 + 13 + 4 + 8  = 32
	\item A - C - D - F - Z = 7 + 13 + 6 + 2 = 28
	\item A - C - D - Z = 7 + 13 + 12 = 32
\end{enumerate}

Ответ: \textbf{28}

\section{№ \textbf{6973}}

\begin{enumerate}
	\item A - B - C - D - E - Z = 4 + 2 + 13 + 4 + 8  = 23 + 8
	\item A - B - C - D - F - Z = 4 + 2 + 13 + 7 + 2 = 26 + 2 = 28
	\item A - B - C - D - Z = 4 + 2 + 13 + 11 = 30
	\item A - B - C - Z = 4 + 2 + 27 = 33
	\item A - C - D - E - Z = 10 + 13 + 4 + 8 =35
\end{enumerate}

Ответ: \textbf{28}

\section{№ \textbf{5261}}

\begin{enumerate}
	\item A - B - C - D - E - F - Z = 4 + 4 + 2 + 3 + 8 + 4 = 25 
	\item A - B - C - D - E - Z = 4 + 4 + 2 + 3 + 9 = 22
	\item A - B - C - D - F - Z = 4 + 4 + 2 + 11 + 4 = 25
	\item A - B - C - D - Z = 4 + 4 + 2 + 15 = 25
	\item A - D - E - Z = 15 + 3 + 9 = 27
	\item A - C - D - E - Z = 10 + 2 + 3 + 9 = 24
\end{enumerate}

Ответ: \textbf{22}

\section{№ \textbf{5473}}

\begin{enumerate}
	\item A-B-D-E-F = 3 + 5 + 5 + 4 = 17
	\item A-B-D-F = 3 + 5 + 11 = 19
	\item A-D-E-F = 7 + 5 + 4 = 16
\end{enumerate}

Ответ: \textbf{16}

\section{№ \textbf{7981}}

\begin{enumerate}
	\item A - C - D - E - F = 4 + 3 + 2 + 5 = 14
	\item A - D - E - F = 8 + 2 + 5 = 15
\end{enumerate}

Ответ: \textbf{14}

\section{№ \textbf{5633}}

\begin{enumerate}
	\item A - B - D - E - F = 1 + 4 + 4 + 3 = 12 
	\item A - D - E - F = 4 + 4 + 3 = 11
	\item A - C - D - E - F = 2 + 3 + 4 + 3 = 12
\end{enumerate}

Ответ: \textbf{11}

\section{№ \textbf{6172}}

\begin{enumerate}
	\item A - B - C - D - E - Z = 4 + 1 + 2 + 4 + 5 = 16
\end{enumerate}

Ответ: \textbf{16}

\section{№ \textbf{5793}}

\begin{enumerate}
	\item A - C - D - E - F = 4 + 2 + 5 + 1 = 12
\end{enumerate}

Ответ: \textbf{12}

\section{№ \textbf{4704}}

\begin{enumerate}
	\item A - B - D - E - F - Z = 7 + 7 + 2 + 2 + 3 = 21
\end{enumerate}

Ответ: \textbf{21}

\section{№ \textbf{4961}}

\begin{enumerate}
	\item A - B - C - D - E - Z = 4 + 3 + 2 + 4 + 4 = 17
	\item A - B - C - F - Z = 4 + 3 + 11 + 2 = 20
\end{enumerate}

Ответ: \textbf{17}

\section{№ \textbf{6442}}

\begin{enumerate}
	\item A - B - D -  F = 2 + 8 + 6 = 16
\end{enumerate}

Ответ: \textbf{16}

\end{document}
