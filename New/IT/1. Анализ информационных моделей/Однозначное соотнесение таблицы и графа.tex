% !TEX TS-program = pdflatex
% !TEX encoding = UTF-8 Unicode

% This is a simple template for a LaTeX document using the "article" class.
% See "book", "report", "letter" for other types of document.

\documentclass[11pt]{article} % use larger type; default would be 10pt

\usepackage[utf8]{inputenc} % set input encoding (not needed with XeLaTeX)
\usepackage[english,russian]{babel}

%%% Examples of Article customizations
% These packages are optional, depending whether you want the features they provide.
% See the LaTeX Companion or other references for full information.

%%% PAGE DIMENSIONS
\usepackage{geometry} % to change the page dimensions
\geometry{a4paper} % or letterpaper (US) or a5paper or....
% \geometry{margin=2in} % for example, change the margins to 2 inches all round
% \geometry{landscape} % set up the page for landscape
%   read geometry.pdf for detailed page layout information

\usepackage{graphicx} % support the \includegraphics command and options

% \usepackage[parfill]{parskip} % Activate to begin paragraphs with an empty line rather than an indent

%%% PACKAGES
\usepackage{booktabs} % for much better looking tables
\usepackage{array} % for better arrays (eg matrices) in maths
\usepackage{paralist} % very flexible & customisable lists (eg. enumerate/itemize, etc.)
\usepackage{verbatim} % adds environment for commenting out blocks of text & for better verbatim
\usepackage{subfig} % make it possible to include more than one captioned figure/table in a single float
% These packages are all incorporated in the memoir class to one degree or another...

%%% HEADERS & FOOTERS
\usepackage{fancyhdr} % This should be set AFTER setting up the page geometry
\pagestyle{fancy} % options: empty , plain , fancy
\renewcommand{\headrulewidth}{0pt} % customise the layout...
\lhead{}\chead{}\rhead{}
\lfoot{}\cfoot{\thepage}\rfoot{}

%%% SECTION TITLE APPEARANCE
\usepackage{sectsty}
\allsectionsfont{\sffamily\mdseries\upshape} % (See the fntguide.pdf for font help)
% (This matches ConTeXt defaults)

%%% ToC (table of contents) APPEARANCE
\usepackage[nottoc,notlof,notlot]{tocbibind} % Put the bibliography in the ToC
\usepackage[titles,subfigure]{tocloft} % Alter the style of the Table of Contents
\renewcommand{\cftsecfont}{\rmfamily\mdseries\upshape}
\renewcommand{\cftsecpagefont}{\rmfamily\mdseries\upshape} % No bold!

%%% END Article customizations

%%% The "real" document content comes below...

\title{Однозначное соотнесение таблицы и графа}
\author{Lisid Lakonsky}
%\date{} % Activate to display a given date or no date (if empty),
         % otherwise the current date is printed 

\begin{document}
\maketitle

\section{№ \textbf{9354}}

\paragraph{А, Б, Д, К - единственные н.п., в которые ведет две дороги. Следовательно, это н.п. П1, П3, П5 и П7.}

\paragraph{Г - единственный н.п., в который ведет три дороги. Следовательно, это н.п. П2.}

\paragraph{Следовательно, В и Е - оставшиеся н.п.: П4 и П6. Расстояние между ними - 20 км.}

\paragraph{Ответ: \textbf{20}}

\section{№ \textbf{9753}}

\paragraph{Г - единственный н.п., в который ведет три дороги. Следовательно, это н.п. П2.}

\paragraph{Е - единственный н.п., в который ведет четыре дороги. Следовательно, это н.п. П4.}

\paragraph{Расстояние между П2 и П4 равняется 40 км.}

\paragraph{Ответ: \textbf{40}}

\section{№ \textbf{9789}}

\paragraph{В - единственный н.п., в который ведет пять дорог. Следовательно, то н.п. П6.}

\paragraph{Г - единственный н.п., в который ведет три дороги. Следовательно, это н.п. П2.}

\paragraph{Расстояние между П2 и П6 равняется 55 км.}

\paragraph{Ответ: \textbf{55}}

\section{№ \textbf{10279}}

\paragraph{Б и Д - два из трех н.п., в которые ведет три дороги.}

\paragraph{В другой населенный пункт с тремя дорогами ведет н.п. с двумя дорогами (П1), следовательно этот н.п. - П2.}

\paragraph{Следовательно, искомые н.п. - П7 и П3.}

\paragraph{Расстояние между ними равно восьми км.}

\paragraph{Ответ: \textbf{8}}

\section{№ \textbf{10377}}

\paragraph{Г - единственный населенный пункт, в который ведет пять дорог. Следовательно, это н.п. П2.}

\paragraph{А, Е, К - н.п. с тремя дорогами. Следовательно, это н.п. П6, П3 и П1 (без порядка).}

\paragraph{Е - единственный н.п. из вышеназванных, в который не ведет Г (П2). То есть, это н.п. П6.}

\paragraph{К - единственный н.п. с тремя дорогами, в который ведет и Е, и Г. То есть, это н.п. П1.}

\paragraph{Оставшийся н.п. - населенный пункт А (П3).}

\paragraph{Расстояние между П3 и П2 равняется 22 км.}

\paragraph{Ответ: \textbf{22}}

\end{document}
