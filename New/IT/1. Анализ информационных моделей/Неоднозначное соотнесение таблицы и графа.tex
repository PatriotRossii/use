% !TEX TS-program = pdflatex
% !TEX encoding = UTF-8 Unicode

% This is a simple template for a LaTeX document using the "article" class.
% See "book", "report", "letter" for other types of document.

\documentclass[11pt]{article} % use larger type; default would be 10pt

\usepackage[utf8]{inputenc} % set input encoding (not needed with XeLaTeX)
\usepackage[english,russian]{babel}

%%% Examples of Article customizations
% These packages are optional, depending whether you want the features they provide.
% See the LaTeX Companion or other references for full information.

%%% PAGE DIMENSIONS
\usepackage{geometry} % to change the page dimensions
\geometry{a4paper} % or letterpaper (US) or a5paper or....
% \geometry{margin=2in} % for example, change the margins to 2 inches all round
% \geometry{landscape} % set up the page for landscape
%   read geometry.pdf for detailed page layout information

\usepackage{graphicx} % support the \includegraphics command and options

% \usepackage[parfill]{parskip} % Activate to begin paragraphs with an empty line rather than an indent

%%% PACKAGES
\usepackage{booktabs} % for much better looking tables
\usepackage{array} % for better arrays (eg matrices) in maths
\usepackage{paralist} % very flexible & customisable lists (eg. enumerate/itemize, etc.)
\usepackage{verbatim} % adds environment for commenting out blocks of text & for better verbatim
\usepackage{subfig} % make it possible to include more than one captioned figure/table in a single float
% These packages are all incorporated in the memoir class to one degree or another...

%%% HEADERS & FOOTERS
\usepackage{fancyhdr} % This should be set AFTER setting up the page geometry
\pagestyle{fancy} % options: empty , plain , fancy
\renewcommand{\headrulewidth}{0pt} % customise the layout...
\lhead{}\chead{}\rhead{}
\lfoot{}\cfoot{\thepage}\rfoot{}

%%% SECTION TITLE APPEARANCE
\usepackage{sectsty}
\allsectionsfont{\sffamily\mdseries\upshape} % (See the fntguide.pdf for font help)
% (This matches ConTeXt defaults)

%%% ToC (table of contents) APPEARANCE
\usepackage[nottoc,notlof,notlot]{tocbibind} % Put the bibliography in the ToC
\usepackage[titles,subfigure]{tocloft} % Alter the style of the Table of Contents
\renewcommand{\cftsecfont}{\rmfamily\mdseries\upshape}
\renewcommand{\cftsecpagefont}{\rmfamily\mdseries\upshape} % No bold!

%%% END Article customizations

%%% The "real" document content comes below...

\title{Неоднозначное соотнесение таблицы и графа}
\author{Lisid Lakonsky}
%\date{} % Activate to display a given date or no date (if empty),
         % otherwise the current date is printed 

\begin{document}
\maketitle

\section{№ \textbf{15619}}

\paragraph{B - 5 единственная точка с пятью путями. Значит, ей соответствует четвертая точка.}

\paragraph{D - единственная точка с тремя путями. Значит, ей соответствует вторая точка.}

\paragraph{C и E - единственные точки, в которые ведут и B, и D. То есть, вторая и четвертая точки. Им соответствуют первая и шестая точки.}

\paragraph{Неопознанными остались лишь искомые точки: третья и пятая.}

\paragraph{Ответ: \textbf{35}}

\section{№ \textbf{15843}}

\paragraph{F - единственная точка с шестью путями. Значит, ей соответствует третья точка.}

\paragraph{C и E - единственные точки с двумя путями. Значит, им соответствуют четвертая и пятая точки.}

\paragraph{B и D - единственные точки, исключая F, в которые есть путь из E и C соответственно. Следовательно, им соответствуют первая и вторая точки.}

\paragraph{Оставшиеся точки - искомые.}

\paragraph{Ответ: \textbf{67}}

\section{№ \textbf{15971}}

\paragraph{Г, Е и А - единственные населенные пункты, в которые ведут две дороги. Следовательно, это населенные пункты П1, П3 и П5.}

\paragraph{Д, К - населенные пункты, в которые ведут одновременно два из них. Следовательно, это населенные пункты П4 и П7.}

\paragraph{Следовательно, искомые населенные пункты - П2 и П6.}

\paragraph{Ответ: \textbf{26}}

\section{№ \textbf{16030}}

\paragraph{A, E и D - единственные н.п., в которые ведут лишь две дороги. Это н.п. один, четыре и семь.}

\paragraph{B и C - н.п., в которые ведут одновременно два из них. Это н.п. два и шесть.}

\paragraph{Ответ: \textbf{26}}

\section{№ \textbf{23901}}

\paragraph{A - единственный н.п., в который ведет четыре дороги. Следовательно, это н.п. П2.}

\paragraph{C и F - единственные н.п., в которые ведет две дороги. Следовательно, это н.п. П1 и П3.}

\paragraph{D и B - единственные н.п., в которые ведет и A, и C или F. Следовательно, это н.п. П6 и П7.}

\paragraph{Оставшиеся н.п. - искомые.}

\paragraph{Ответ: \textbf{45}}

\end{document}
